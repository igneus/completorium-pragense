\documentclass[a5paper, 12pt]{book}
\usepackage[latin]{babel}
\usepackage[left=2cm, right=2cm, top=2cm, bottom=2cm]{geometry}

\usepackage{fontspec}
\usepackage[hidelinks]{hyperref}
\usepackage{etoolbox}
\usepackage[show]{ed}

% for gregorio
\usepackage{luatextra}
\usepackage{graphicx}
\usepackage{gregoriotex}

\usepackage[
  backend=biber,
  style=iso-authoryear,
  sortlocale=cs_CZ,
  maxnames=3,
  firstinits=true,
]{biblatex}

\bibliography{biblio}

% title of a formulary
\newcommand{\formulary}[1]{\chapter{#1}}

% rubric
\newcommand{\rubrica}[1]{\textit{#1}}

% text incipit inside a rubric
\newcommand{\incipit}[1]{\textup{#1}}

% insert full text of a hymn
\newcommand{\hymnus}[1]{%
  \phantomsection
  \label{hym:#1}
  \input{hymni/#1}}

% insert chant notation
\newcommand{\cantus}[1]{%
  \phantomsection
  \label{can:#1}
  \input{cantus/#1.gtex}}

% inline page reference to full text or notation elsewhere
\newcommand{\quickref}[1]{\quickrefFormat{\pageref{#1}}}
\newcommand{\missingref}{\quickrefFormat{?}}
\newcommand{\quickrefFormat}[1]{\textsuperscript{\textup{$\rightarrow$#1}}}

% text not literally reproducing the source,
% but supplemented by the editor
\newcommand{\editorial}[1]{[#1]}





\title{Completorium secundum Rubricam pragensem}

\begin{document}

\maketitle

\chapter{Psalmi}
\label{psalmi}

\edissue{add psalm texts}

\chapter{Hymnarius}

\hymnus{conditor}
\edissue{notation}

\hymnus{corde}
\edissue{notation}

\chapter{Preces}
\label{preces}

Kyrieleison. Kyrieleison. Kyrieleison.\\
Christeleison. Christeleison. Christeleison.\\
Kyrieleison. Kyrieleison. Kyrieleison.\edissue{compare other contemporary local sources how kyrie was most usually written}
\\
Pater noster. Et ne nos.\\
In pace in idipsum. Dormiam et requiescam.\\
Credo in Deum. Carnis resurrectionem. Et vitam.
\edissue{Clarify the execution, provide full texts.}
\\
Benedicamus Patrem et Filium cum Sancto Spiritu.
Laudemus et superexaltemus sum in saecula.\\
Benedictus es, Domine, in firmamento coeli.
Et laudabilis et gloriosus et superexaltatus in saecula.\\
Benedicat nos omnipotens Dominus. Amen.\edissue{uncertain}
\\
Dignare Domine nocte ista.
Sine peccato nos custodire.\\
Miserere nostri, Domine, miserere nostri.\\
Fiat misericordia tua, Domine, super nos.
Quemadmodum speravimus in te.\\
Sacerdotes tui induantur iustitia.\edissue{verify exact spelling, this one differs from now standard Vulgate}
Et sancti tui exsultent.

\noindent\rubrica{\editorial{Versus proprius tempori aut festo hic adnectitur.}}

\noindent{}Domine exaudi orationem meam.
Et clamor meus \editorial{ad te veniat}.\\
Dominus vobiscum.\edissue{This dialogue is probably a standard introduction for the prayer, not part of the preces as such.}
\editorial{Et cum spiritu tuo.}

\footcite[1v]{xiv_a_19}
\edissue{Try to find in printed breviaries -- so far no luck}

\formulary{In Adventu Domini}

\rubrica{Hymnus \incipit{Conditor alme siderum}\quickref{hym:conditor}
  in completorio
  nunquam mutatur, nisi in Conceptione sanctae Mariae.}\footcite[83r]{bp1502}
\vspace{3mm}

\rubrica{Ad completorium psalmi consueti,\quickref{psalmi}
  quos sequitur \incipit{alleluia.}}
\edissue{Not sure what exactly the alleluia means. An antiphon super psalmos? But XIV A 19 doesn't notate it.}

\rubrica{Capitulum}
Patientes estote et confirmate corda vestra:
quoniam adventus Domini appropinquabit.
Deo \editorial{gratias}.

\rubrica{Hymnus}
Conditor alme.\quickref{hym:conditor}

\rubrica{V.}
Emitte agnum Domine, dominatorem terrae.
De petra deserti ad montem filiae Sion.

\rubrica{Ad Nunc dimittis antiphona}
Qui venturus est veniet et non tardabit. Iam non erit timor in finibus nostris.\quickref{can:adventus_quiventurus}

\rubrica{Preces solitae\quickref{preces}
  dicuntur versum adiungendo de Adventu, scilicet istum.}
Veni Domine et noli tardare. Relaxa facinora plebis tuae.

\rubrica{Oratio}
Fac nos quaesumus Domine Deus noster pervigiles atque solicitos adventum
Christi Filii tui exspectare: ut dum venerit pulsans,
non dormientes in peccatis, sed vigilantes in suis laudibus inveniat
exsultantes. Qui tecum.

\rubrica{Alia oratio}
Prope esto Domine omnibus exspectantibus te in veritate:
ut in adventu Domini nostri Iesu Christi placitis tibi actibus praesentemur.
Qui tecum.

\rubrica{Una istarum orationum dicatur quae placet et non plus.
  Hic ordo completorii non mutatur per totum Adventum
  nisi in Conceptione sanctae Mariae.}\footcite[83r-83v]{bp1502}

\headingCantus

\cantus{adventus_quiventurus}

\formulary{In Sacra nocte}

\rubrica{Ad completorium psalmi consueti,\quickref{psalmi}
  quos sequitur \incipit{alleluia.}}
\edissue{repeated verbatim from Advent}

\rubrica{Capitulum, hymnus, versus et preces non dicuntur,
  sed statim antiphona ad Nunc dimittis subiungitur.
  Et illud completorium canitur in media ecclesia\edissue{cite studies on the topography of the Prague cathedral}
  hora XXIII sive XXIIII.}

\rubrica{Antiphona}
Dum ortus fuerit sol de coelo, videbitis regem regum
procedentem a patre tanquam sponsum de thalamo suo.\quickref{can:nox_sacra_dumortus}

\rubrica{Oratio}
Deus qui hanc sacratissimam noctem veri luminis fecisti
illustratione clarescere:
da quaesumus: ut cuius mysterium in terra cognovimus:
eius quoque gaudiis perfruamur.
Qui tecum vivit.\footcite[98r]{bp1502}

\section*{Cantus}

\cantus{nox_sacra_dumortus}

\chapter{In die sancto Nativitatis Christi}

\rubrica{Ad completorium super psalmos consuetos\quickref{psalmi}
  antiphona}
Verbum caro factum est, et habitabit\edissue{sic; compare other sources} in nobis,
et vidimus gloriam eius, gloriam quasi Unigeniti a Patre.

\rubrica{Capitulum}
Parvulus natus est nobis: et filius datus est nobis.
Deo gratias.

\rubrica{Hymnus}
Corde natus.\quickref{hym:corde}

\rubrica{V.} Benedictus qui.\edissue{expand}

\rubrica{Ad Nunc dimittis antiphona}
Magnum nomen Domini Emanuel, quod annuntiatum est per Gabriel,
hodie apparuit in Israel.
Per Mariam Virginem natus est Rex.\missingref{}

\rubrica{Preces consuetas\quickref{preces}
  dicimus, versum in fine adiungendo de Nativitate Christi.\edissue{Provide the verse}}

\rubrica{Oratio}
Praesta, quaesumus, omnipotens Deus:
ut natus hodie Salvator mundi:
sicut divinae generationis est auctor:
ita et immortalitatis sit ipse largitor.
Qui cum.

\rubrica{Hoc completorium non mutatur usque ad Octavam Epiphaniae,
  praeter Vigiliam Circumcisionis
  et festum eius.}\footcite[101r]{bp1502}

\formulary{In Circumcisione Domini}

\rubrica{\editorial{Post primam vesperam}
  ad completorium omnia ut supra.\quickref{form:nativitas}}

\rubrica{Ad Nunc dimittis antiphona}
Glorificamus te Dei Genitrix, quia ex te natus est Christus,
salva omnes qui te glorificant.\missingref{}
\edissue{XV A 10 64r just a rubric with text incipit}

\rubrica{quae durat usque ad Epiphaniam Domini.}

\rubrica{Oratio \incipit{Deus qui salutis}\edissue{find the collect}
  hec sola.}\footcite[109v]{bp1502}

\rubrica{\editorial{Post secundas vesperas}
  ad completorium omnia ut sero.}\footcite[110v]{bp1502}


\printbibliography

\tableofcontents

\end{document}
