\chapter{In Sacra nocte}

\rubrica{Ad completorium psalmi consueti,\quickref{psalmi}
  quos sequitur \incipit{alleluia.}}
\edissue{repeated verbatim from Advent}

\rubrica{Capitulum, hymnus, versus et preces non dicuntur,
  sed statim antiphona ad Nunc dimittis subiungitur.
  Et illud completorium canitur in media ecclesia\edissue{cite studies on the topography of the Prague cathedral}
  hora XXIII sive XXIIII.}

\rubrica{Antiphona}
Dum ortus fuerit sol de coelo, videbitis regem regum
procedentem a patre tanquam sponsum de thalamo suo.\quickref{can:nox_sacra_dumortus}

\rubrica{Oratio}
Deus qui hanc sacratissimam noctem veri luminis fecisti
illustratione clarescere:
da quaesumus: ut cuius mysterium in terra cognovimus:
eius quoque gaudiis perfruamur.
Qui tecum vivit.\footcite[98r]{bp1502}

\section*{Cantus}

\cantus{nox_sacra_dumortus}
\edissue{prefer more canonical source: CZ-Pn XV A 10, 41r}
