\formulary{In Adventu Domini}

\rubrica{Hymnus \incipit{Conditor alme siderum}\quickref{hym:conditor}
  in completorio
  nunquam mutatur, nisi in Conceptione sanctae Mariae.}\footcite[83r]{bp1502}
\vspace{3mm}

\rubrica{Ad completorium psalmi consueti,\quickref{psalmi}
  quos sequitur \incipit{alleluia.}}
\edissue{Not sure what exactly the alleluia means. An antiphon super psalmos? But XIV A 19 doesn't notate it.}

\rubrica{Capitulum}
Patientes estote et confirmate corda vestra:
quoniam adventus Domini appropinquabit.
Deo \editorial{gratias}.

\rubrica{Hymnus}
Conditor alme.\quickref{hym:conditor}

\rubrica{V.}
Emitte agnum Domine, dominatorem terrae.
De petra deserti ad montem filiae Sion.

\rubrica{Ad Nunc dimittis antiphona}
Qui venturus est veniet et non tardabit. Iam non erit timor in finibus nostris.\quickref{can:adventus_quiventurus}

\rubrica{Preces solitae\quickref{preces}
  dicuntur versum adiungendo de Adventu, scilicet istum.}
Veni Domine et noli tardare. Relaxa facinora plebis tuae.

\rubrica{Oratio}
Fac nos quaesumus Domine Deus noster pervigiles atque solicitos adventum
Christi Filii tui exspectare: ut dum venerit pulsans,
non dormientes in peccatis, sed vigilantes in suis laudibus inveniat
exsultantes. Qui tecum.

\rubrica{Alia oratio}
Prope esto Domine omnibus exspectantibus te in veritate:
ut in adventu Domini nostri Iesu Christi placitis tibi actibus praesentemur.
Qui tecum.

\rubrica{Una istarum orationum dicatur quae placet et non plus.
  Hic ordo completorii non mutatur per totum Adventum
  nisi in Conceptione sanctae Mariae.}\footcite[83r-83v]{bp1502}

\headingCantus

\cantus{adventus_quiventurus}
