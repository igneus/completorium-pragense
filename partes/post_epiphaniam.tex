\formulary{Post Octavam Epiphaniae}

\rubrica{Ad completorium psalmi consueti\quickref{psalmi}
  quos sequitur \incipit{Alleluia.}}

\rubrica{Capitulum}
Pacem et veritatem diligite: ait Dominus omnipotens.
Deo \editorial{gratias}.

\rubrica{Hymnus} Salvator mundi.\quickref{hym:salvator_mundi}

\rubrica{V.} Custodi nos Domine.\edissue{Provide full text}

\rubrica{Ad Nunc dimittis antiphona}
Salva nos, Domine, vigilantes,
custodi nos dormientes,
ut vigilemus in Christo
et requiescamus in pace.\quickref{can:post_epiphaniam_salva1}

\rubrica{Preces solitae dicuntur.}

\rubrica{Sequuntur orationes.}
Vigila super nos aeterne Salvator,
ne nos apprehendat ille callidus tentator:
quia tu nobis factus es sempiternus adiutor.

\rubrica{Alia}
Visita quaesumus Domine habitationem istam:
et omnes insidias inimici ab ea longe repelle:
et angeli sancti tui in ea habitantes:
nos in pace custodiant:
et benedictio tua sit semper nobiscum.

\rubrica{Alia}
Salvator mundi, salva nos in hac nocte et in omni tempore:
et intercedente pro nobis beata Dei Genitrice et Virgine Maria
cum omnibus sanctis tuis:
et pacem tuam nostris concede temporibus:
et omnia mala a nobis propiciatus exclude.
Qui cum.

\rubrica{Supra notatum capitulum et hymnus
  \incipit{Te lucis,}\quickref{hym:te_lucis}
  versus et antiphonae et orationes dicuntur ad completorium
  per circulum anni,
  praeter Quadragesimam et Tempus paschale et festa
  Beatae Virginis et Adventum Domini et festa Christi.
  Hymnus vero \incipit{Salvator mundi}\quickref{hym:salvator_mundi}
  diebus sabbatinis dicitur, quando feria die dominico canitur,
  exceptis solemnitatibus.}\footcite[121r]{bp1502}

\section{Cantus}

\cantus{post_epiphaniam_salva1}

\rubrica{Alia nota super eandem antiphonam,
  quae canitur in magnis festis per circulum anni.}\footcite[81r]{xv_a_10}

\cantus{post_epiphaniam_salva2}
