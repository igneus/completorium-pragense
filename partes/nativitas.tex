\formulary{In die sancto Nativitatis Christi}

\rubrica{Ad completorium super psalmos consuetos\quickref{psalmi}
  antiphona}
Verbum caro factum est, et habitabit\edissue{
  sic; compare other sources:
  XV A 10 habitavit}
in nobis,
et vidimus gloriam eius, gloriam quasi Unigeniti a Patre.\quickref{can:nativitas_verbumcaro}

\rubrica{Capitulum}
Parvulus natus est nobis: et filius datus est nobis.
Deo gratias.

\rubrica{Hymnus}
Corde natus.\quickref{hym:corde}

\rubrica{V.} Benedictus qui.\edissue{expand}

\rubrica{Ad Nunc dimittis antiphona}
Magnum nomen Domini Emanuel, quod annuntiatum est per Gabriel,
hodie apparuit in Israel.
Per Mariam Virginem natus est Rex.\quickref{can:nativitas_magnum}

\rubrica{Preces consuetas\quickref{preces}
  dicimus, versum in fine adiungendo de Nativitate Christi.\edissue{Provide the verse}}

\rubrica{Oratio}
Praesta, quaesumus, omnipotens Deus:
ut natus hodie Salvator mundi:
sicut divinae generationis est auctor:
ita et immortalitatis sit ipse largitor.
Qui cum.

\rubrica{Hoc completorium non mutatur usque ad Octavam Epiphaniae,
  praeter Vigiliam Circumcisionis
  et festum eius.}\footcite[101r]{bp1502}

\section*{Cantus}

\cantus{nativitas_verbumcaro}

\cantus{nativitas_magnum}

\edissue{XV A 10 has an additional trope \emph{Resonet in laudibus}
  and another antiphon \emph{Verbum caro},
  both sung after Nunc dimittis.
  Do we want to take content like this, not accepted in the later
  printed breviaries which we follow, in account somehow?}
